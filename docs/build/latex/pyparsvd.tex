%% Generated by Sphinx.
\def\sphinxdocclass{report}
\documentclass[letterpaper,10pt,english]{sphinxmanual}
\ifdefined\pdfpxdimen
   \let\sphinxpxdimen\pdfpxdimen\else\newdimen\sphinxpxdimen
\fi \sphinxpxdimen=.75bp\relax

\PassOptionsToPackage{warn}{textcomp}
\usepackage[utf8]{inputenc}
\ifdefined\DeclareUnicodeCharacter
% support both utf8 and utf8x syntaxes
  \ifdefined\DeclareUnicodeCharacterAsOptional
    \def\sphinxDUC#1{\DeclareUnicodeCharacter{"#1}}
  \else
    \let\sphinxDUC\DeclareUnicodeCharacter
  \fi
  \sphinxDUC{00A0}{\nobreakspace}
  \sphinxDUC{2500}{\sphinxunichar{2500}}
  \sphinxDUC{2502}{\sphinxunichar{2502}}
  \sphinxDUC{2514}{\sphinxunichar{2514}}
  \sphinxDUC{251C}{\sphinxunichar{251C}}
  \sphinxDUC{2572}{\textbackslash}
\fi
\usepackage{cmap}
\usepackage[T1]{fontenc}
\usepackage{amsmath,amssymb,amstext}
\usepackage{babel}



\usepackage{times}
\expandafter\ifx\csname T@LGR\endcsname\relax
\else
% LGR was declared as font encoding
  \substitutefont{LGR}{\rmdefault}{cmr}
  \substitutefont{LGR}{\sfdefault}{cmss}
  \substitutefont{LGR}{\ttdefault}{cmtt}
\fi
\expandafter\ifx\csname T@X2\endcsname\relax
  \expandafter\ifx\csname T@T2A\endcsname\relax
  \else
  % T2A was declared as font encoding
    \substitutefont{T2A}{\rmdefault}{cmr}
    \substitutefont{T2A}{\sfdefault}{cmss}
    \substitutefont{T2A}{\ttdefault}{cmtt}
  \fi
\else
% X2 was declared as font encoding
  \substitutefont{X2}{\rmdefault}{cmr}
  \substitutefont{X2}{\sfdefault}{cmss}
  \substitutefont{X2}{\ttdefault}{cmtt}
\fi


\usepackage[Bjarne]{fncychap}
\usepackage{sphinx}

\fvset{fontsize=\small}
\usepackage{geometry}


% Include hyperref last.
\usepackage{hyperref}
% Fix anchor placement for figures with captions.
\usepackage{hypcap}% it must be loaded after hyperref.
% Set up styles of URL: it should be placed after hyperref.
\urlstyle{same}

\addto\captionsenglish{\renewcommand{\contentsname}{Contents:}}

\usepackage{sphinxmessages}
\setcounter{tocdepth}{1}



\title{pyparsvd Documentation}
\date{Mar 24, 2021}
\release{0.0.1}
\author{PyParSVD contributors}
\newcommand{\sphinxlogo}{\vbox{}}
\renewcommand{\releasename}{Release}
\makeindex
\begin{document}

\pagestyle{empty}
\sphinxmaketitle
\pagestyle{plain}
\sphinxtableofcontents
\pagestyle{normal}
\phantomsection\label{\detokenize{index::doc}}

\begin{itemize}
\item {} 
\sphinxAtStartPar
The \sphinxstylestrong{GitHub repository} of this package can be found at \sphinxhref{https://github.com/Romit-Maulik/PyParSVD}{PyParSVD} along installation instructions, and how to get started.

\item {} 
\sphinxAtStartPar
\sphinxstylestrong{Tutorials} can be found at \sphinxhref{https://github.com/Romit-Maulik/PyParSVD/tree/main/tutorials}{PyParSVD\sphinxhyphen{}Tutorials}

\item {} 
\sphinxAtStartPar
The package uses \sphinxhref{https://travis-ci.com}{Travis\sphinxhyphen{}CI} for \sphinxstylestrong{continuous integration}.

\end{itemize}


\chapter{Summary}
\label{\detokenize{index:summary}}
\sphinxAtStartPar
The \sphinxtitleref{PyParSVD} library implements both a serial and a parallel singular value decomposition (SVD).
The implementation of the library is conveniently:
\sphinxhyphen{} Distributed using MPI4Py (for parallel SVD);
\sphinxhyphen{} Streaming \sphinxhyphen{} data can be shown in batches to update the left singular vectors;
\sphinxhyphen{} Randomized \sphinxhyphen{} further acceleration of any serial components of the overall algorithm.

\sphinxAtStartPar
The \sphinxtitleref{PyParSVD} library is organized following an abstract factory design pattern, where we define
a base class in \sphinxtitleref{parsvd\_base.py}, called \sphinxtitleref{ParSVD\_Base} {\hyperref[\detokenize{index:parsvd-base-class}]{\sphinxcrossref{\DUrole{std,std-ref}{ParSVD base class}}}}, that implements
functions and parameters available to all derived classes. In addition, it implements two abstract
functions, \sphinxtitleref{initialize()} and \sphinxtitleref{incorporate\_data()} which implementation must be provided by the
derived classes.
\begin{description}
\item[{The classes derived from the base class are the following:}] \leavevmode\begin{itemize}
\item {} 
\sphinxAtStartPar
\sphinxtitleref{ParSVD\_serial} (implemented in \sphinxtitleref{parsvd\_serial.py}) {\hyperref[\detokenize{index:module-pyparsvd.parsvd_serial}]{\sphinxcrossref{\DUrole{std,std-ref}{ParSVD serial}}}}

\item {} 
\sphinxAtStartPar
\sphinxtitleref{ParSVD\_parallel} (implemented in \sphinxtitleref{parsvd\_parallel.py}) {\hyperref[\detokenize{index:module-pyparsvd.parsvd_parallel}]{\sphinxcrossref{\DUrole{std,std-ref}{ParSVD parallel}}}}

\end{itemize}

\end{description}

\sphinxAtStartPar
These derived classes contain the actual implementation of two different different versions
of SVD algorithms, one that is serial (\sphinxtitleref{parsvd\_serial.py}) and one that is parallel (\sphinxtitleref{parsvd\_parallel.py}).

\sphinxAtStartPar
\sphinxstylestrong{It should be noted that the design pattern chosen allows for the
easy addition of derived classes that can implement a new SVD versions.}

\sphinxAtStartPar
The distributed (parallel) implementation of the SVD in \sphinxtitleref{parsvd\_parallel.py} follows
\sphinxhref{https://www.sciencedirect.com/science/article/pii/S0377042715005774}{(Wang et al. 2016)}.
The streaming algorithm used in the library for both \sphinxtitleref{parsvd\_serial.py} and \sphinxtitleref{parsvd\_parallel.py}
is from \sphinxhref{https://ieeexplore.ieee.org/abstract/document/723422}{(Levy and Lindenbaum 1998)},
where the parallel QR algorithm (the TSQR method) required for the streaming feature follows
\sphinxhref{https://ieeexplore.ieee.org/document/6691583}{(Benson et al. 2013)}.
Finally, the randomized algorithm adopted in the library follows
\sphinxhref{https://epubs.siam.org/doi/abs/10.1137/090771806}{(Halko et al 2013)}.

\sphinxAtStartPar
Additionally to these modules, we also provide some post\sphinxhyphen{}processing capabilities
to visualize the results. These are implemented in \sphinxtitleref{postprocessing.py} {\hyperref[\detokenize{index:postprocessing-module}]{\sphinxcrossref{\DUrole{std,std-ref}{Postprocessing module}}}}.
The functions in post\sphinxhyphen{}processing can be accessed directly from the base class, and in particular
from the \sphinxtitleref{ParSVD object} returned by the \sphinxtitleref{initialize()} and \sphinxtitleref{incorporate\_data()} function.
They can also be accessed separately from the base class, as the post\sphinxhyphen{}processing module
constitutes a standalone module. In practice, once you run an analysis, you can load the
results at a later stage and use the post\sphinxhyphen{}processing module to visualize the results or
you can implement you own visualization tools, that best suit your needs.


\section{Indices and table}
\label{\detokenize{index:indices-and-table}}\begin{itemize}
\item {} 
\sphinxAtStartPar
\DUrole{xref,std,std-ref}{genindex}

\item {} 
\sphinxAtStartPar
\DUrole{xref,std,std-ref}{modindex}

\end{itemize}


\chapter{ParSVD main modules}
\label{\detokenize{index:parsvd-main-modules}}
\sphinxAtStartPar
The ParSVD main modules constitutes the backbone of the \sphinxtitleref{PyParSVD} library.
They are constituted by the base class:
\begin{itemize}
\item {} 
\sphinxAtStartPar
\sphinxtitleref{ParSVD\_Base} (implemented in \sphinxtitleref{parsvd\_base.py}) {\hyperref[\detokenize{index:parsvd-base-class}]{\sphinxcrossref{\DUrole{std,std-ref}{ParSVD base class}}}}

\end{itemize}

\sphinxAtStartPar
along with its derived classes:
\begin{itemize}
\item {} 
\sphinxAtStartPar
\sphinxtitleref{ParSVD\_serial} (implemented in \sphinxtitleref{parsvd\_serial.py}) {\hyperref[\detokenize{index:module-pyparsvd.parsvd_serial}]{\sphinxcrossref{\DUrole{std,std-ref}{ParSVD serial}}}}

\item {} 
\sphinxAtStartPar
\sphinxtitleref{ParSVD\_parallel} (implemented in \sphinxtitleref{parsvd\_parallel.py}) {\hyperref[\detokenize{index:module-pyparsvd.parsvd_parallel}]{\sphinxcrossref{\DUrole{std,std-ref}{ParSVD parallel}}}}

\end{itemize}


\section{ParSVD base class}
\label{\detokenize{index:parsvd-base-class}}
\sphinxAtStartPar
The \sphinxstylestrong{ParSVD base class} is intended to hold functions that are shared
by all derived classes. It follows an abstract factory design pattern.

\phantomsection\label{\detokenize{index:module-pyparsvd.parsvd_base}}\index{module@\spxentry{module}!pyparsvd.parsvd\_base@\spxentry{pyparsvd.parsvd\_base}}\index{pyparsvd.parsvd\_base@\spxentry{pyparsvd.parsvd\_base}!module@\spxentry{module}}\index{ParSVD\_Base (class in pyparsvd.parsvd\_base)@\spxentry{ParSVD\_Base}\spxextra{class in pyparsvd.parsvd\_base}}

\begin{fulllineitems}
\phantomsection\label{\detokenize{index:pyparsvd.parsvd_base.ParSVD_Base}}\pysiglinewithargsret{\sphinxbfcode{\sphinxupquote{class }}\sphinxbfcode{\sphinxupquote{ParSVD\_Base}}}{\emph{\DUrole{n}{K}}, \emph{\DUrole{n}{ff}}, \emph{\DUrole{n}{low\_rank}\DUrole{o}{=}\DUrole{default_value}{False}}, \emph{\DUrole{n}{results\_dir}\DUrole{o}{=}\DUrole{default_value}{\textquotesingle{}results\textquotesingle{}}}}{}
\sphinxAtStartPar
PyParSVD base class. It implements data and methods shared
across the derived classes.
\begin{quote}\begin{description}
\item[{Parameters}] \leavevmode\begin{itemize}
\item {} 
\sphinxAtStartPar
\sphinxstyleliteralstrong{\sphinxupquote{K}} (\sphinxhref{https://docs.python.org/2/library/functions.html\#int}{\sphinxstyleliteralemphasis{\sphinxupquote{int}}}) \textendash{} number of modes to truncate.

\item {} 
\sphinxAtStartPar
\sphinxstyleliteralstrong{\sphinxupquote{ff}} (\sphinxhref{https://docs.python.org/2/library/functions.html\#int}{\sphinxstyleliteralemphasis{\sphinxupquote{int}}}) \textendash{} forget factor.

\item {} 
\sphinxAtStartPar
\sphinxstyleliteralstrong{\sphinxupquote{low\_rank}} (\sphinxhref{https://docs.python.org/2/library/functions.html\#bool}{\sphinxstyleliteralemphasis{\sphinxupquote{bool}}}) \textendash{} if True, it uses a low rank algorithm to speed up computations.

\item {} 
\sphinxAtStartPar
\sphinxstyleliteralstrong{\sphinxupquote{results\_dir}} (\sphinxhref{https://docs.python.org/2/library/functions.html\#str}{\sphinxstyleliteralemphasis{\sphinxupquote{str}}}) \textendash{} if specified, it saves the results in \sphinxtitleref{results\_dir}.            Default save path is under a folder called \sphinxtitleref{results} in current working path.

\end{itemize}

\end{description}\end{quote}
\index{K() (ParSVD\_Base property)@\spxentry{K()}\spxextra{ParSVD\_Base property}}

\begin{fulllineitems}
\phantomsection\label{\detokenize{index:pyparsvd.parsvd_base.ParSVD_Base.K}}\pysigline{\sphinxbfcode{\sphinxupquote{property }}\sphinxbfcode{\sphinxupquote{K}}}
\sphinxAtStartPar
Get the number of modes to truncate.
\begin{quote}\begin{description}
\item[{Returns}] \leavevmode
\sphinxAtStartPar
number of modes to truncate.

\item[{Return type}] \leavevmode
\sphinxAtStartPar
\sphinxhref{https://docs.python.org/2/library/functions.html\#int}{int}

\end{description}\end{quote}

\end{fulllineitems}

\index{comm() (ParSVD\_Base property)@\spxentry{comm()}\spxextra{ParSVD\_Base property}}

\begin{fulllineitems}
\phantomsection\label{\detokenize{index:pyparsvd.parsvd_base.ParSVD_Base.comm}}\pysigline{\sphinxbfcode{\sphinxupquote{property }}\sphinxbfcode{\sphinxupquote{comm}}}
\sphinxAtStartPar
Get the parallel MPI Communicator.
\begin{quote}\begin{description}
\item[{Returns}] \leavevmode
\sphinxAtStartPar
comm.

\item[{Return type}] \leavevmode
\sphinxAtStartPar
MPI\_Comm

\end{description}\end{quote}

\end{fulllineitems}

\index{ff() (ParSVD\_Base property)@\spxentry{ff()}\spxextra{ParSVD\_Base property}}

\begin{fulllineitems}
\phantomsection\label{\detokenize{index:pyparsvd.parsvd_base.ParSVD_Base.ff}}\pysigline{\sphinxbfcode{\sphinxupquote{property }}\sphinxbfcode{\sphinxupquote{ff}}}
\sphinxAtStartPar
Get the forget factor.
\begin{quote}\begin{description}
\item[{Returns}] \leavevmode
\sphinxAtStartPar
forget factor.

\item[{Return type}] \leavevmode
\sphinxAtStartPar
\sphinxhref{https://docs.python.org/2/library/functions.html\#int}{int}

\end{description}\end{quote}

\end{fulllineitems}

\index{iteration() (ParSVD\_Base property)@\spxentry{iteration()}\spxextra{ParSVD\_Base property}}

\begin{fulllineitems}
\phantomsection\label{\detokenize{index:pyparsvd.parsvd_base.ParSVD_Base.iteration}}\pysigline{\sphinxbfcode{\sphinxupquote{property }}\sphinxbfcode{\sphinxupquote{iteration}}}
\sphinxAtStartPar
Get the number of data incorporation performed          in the streaming data ingestion.
\begin{quote}\begin{description}
\item[{Returns}] \leavevmode
\sphinxAtStartPar
iterations.

\item[{Return type}] \leavevmode
\sphinxAtStartPar
\sphinxhref{https://docs.python.org/2/library/functions.html\#int}{int}

\end{description}\end{quote}

\end{fulllineitems}

\index{low\_rank() (ParSVD\_Base property)@\spxentry{low\_rank()}\spxextra{ParSVD\_Base property}}

\begin{fulllineitems}
\phantomsection\label{\detokenize{index:pyparsvd.parsvd_base.ParSVD_Base.low_rank}}\pysigline{\sphinxbfcode{\sphinxupquote{property }}\sphinxbfcode{\sphinxupquote{low\_rank}}}
\sphinxAtStartPar
Get the low rank behaviour.
\begin{quote}\begin{description}
\item[{Returns}] \leavevmode
\sphinxAtStartPar
low rank behaviour.

\item[{Return type}] \leavevmode
\sphinxAtStartPar
\sphinxhref{https://docs.python.org/2/library/functions.html\#bool}{bool}

\end{description}\end{quote}

\end{fulllineitems}

\index{modes() (ParSVD\_Base property)@\spxentry{modes()}\spxextra{ParSVD\_Base property}}

\begin{fulllineitems}
\phantomsection\label{\detokenize{index:pyparsvd.parsvd_base.ParSVD_Base.modes}}\pysigline{\sphinxbfcode{\sphinxupquote{property }}\sphinxbfcode{\sphinxupquote{modes}}}
\sphinxAtStartPar
Get the modes.
\begin{quote}\begin{description}
\item[{Returns}] \leavevmode
\sphinxAtStartPar
modes.

\item[{Return type}] \leavevmode
\sphinxAtStartPar
ndarray

\end{description}\end{quote}

\end{fulllineitems}

\index{n\_modes() (ParSVD\_Base property)@\spxentry{n\_modes()}\spxextra{ParSVD\_Base property}}

\begin{fulllineitems}
\phantomsection\label{\detokenize{index:pyparsvd.parsvd_base.ParSVD_Base.n_modes}}\pysigline{\sphinxbfcode{\sphinxupquote{property }}\sphinxbfcode{\sphinxupquote{n\_modes}}}
\sphinxAtStartPar
Get the number of modes.
\begin{quote}\begin{description}
\item[{Returns}] \leavevmode
\sphinxAtStartPar
number of modes.

\item[{Return type}] \leavevmode
\sphinxAtStartPar
\sphinxhref{https://docs.python.org/2/library/functions.html\#int}{int}

\end{description}\end{quote}

\end{fulllineitems}

\index{nprocs() (ParSVD\_Base property)@\spxentry{nprocs()}\spxextra{ParSVD\_Base property}}

\begin{fulllineitems}
\phantomsection\label{\detokenize{index:pyparsvd.parsvd_base.ParSVD_Base.nprocs}}\pysigline{\sphinxbfcode{\sphinxupquote{property }}\sphinxbfcode{\sphinxupquote{nprocs}}}
\sphinxAtStartPar
Get the number processors
\begin{quote}\begin{description}
\item[{Returns}] \leavevmode
\sphinxAtStartPar
processors.

\item[{Return type}] \leavevmode
\sphinxAtStartPar
\sphinxhref{https://docs.python.org/2/library/functions.html\#int}{int}

\end{description}\end{quote}

\end{fulllineitems}

\index{plot\_1D\_modes() (ParSVD\_Base method)@\spxentry{plot\_1D\_modes()}\spxextra{ParSVD\_Base method}}

\begin{fulllineitems}
\phantomsection\label{\detokenize{index:pyparsvd.parsvd_base.ParSVD_Base.plot_1D_modes}}\pysiglinewithargsret{\sphinxbfcode{\sphinxupquote{plot\_1D\_modes}}}{\emph{\DUrole{n}{idxs}\DUrole{o}{=}\DUrole{default_value}{{[}0{]}}}, \emph{\DUrole{n}{title}\DUrole{o}{=}\DUrole{default_value}{\textquotesingle{}\textquotesingle{}}}, \emph{\DUrole{n}{figsize}\DUrole{o}{=}\DUrole{default_value}{(12, 8)}}, \emph{\DUrole{n}{filename}\DUrole{o}{=}\DUrole{default_value}{None}}}{}
\sphinxAtStartPar
See method implementation in the postprocessing module.

\end{fulllineitems}

\index{plot\_singular\_values() (ParSVD\_Base method)@\spxentry{plot\_singular\_values()}\spxextra{ParSVD\_Base method}}

\begin{fulllineitems}
\phantomsection\label{\detokenize{index:pyparsvd.parsvd_base.ParSVD_Base.plot_singular_values}}\pysiglinewithargsret{\sphinxbfcode{\sphinxupquote{plot\_singular\_values}}}{\emph{\DUrole{n}{idxs}\DUrole{o}{=}\DUrole{default_value}{{[}0{]}}}, \emph{\DUrole{n}{title}\DUrole{o}{=}\DUrole{default_value}{\textquotesingle{}\textquotesingle{}}}, \emph{\DUrole{n}{figsize}\DUrole{o}{=}\DUrole{default_value}{(12, 8)}}, \emph{\DUrole{n}{filename}\DUrole{o}{=}\DUrole{default_value}{None}}}{}
\sphinxAtStartPar
See method implementation in the postprocessing module.

\end{fulllineitems}

\index{rank() (ParSVD\_Base property)@\spxentry{rank()}\spxextra{ParSVD\_Base property}}

\begin{fulllineitems}
\phantomsection\label{\detokenize{index:pyparsvd.parsvd_base.ParSVD_Base.rank}}\pysigline{\sphinxbfcode{\sphinxupquote{property }}\sphinxbfcode{\sphinxupquote{rank}}}
\sphinxAtStartPar
Get the parallel MPI Rank.
\begin{quote}\begin{description}
\item[{Returns}] \leavevmode
\sphinxAtStartPar
rank.

\item[{Return type}] \leavevmode
\sphinxAtStartPar
MPI\_Rank

\end{description}\end{quote}

\end{fulllineitems}

\index{singular\_values() (ParSVD\_Base property)@\spxentry{singular\_values()}\spxextra{ParSVD\_Base property}}

\begin{fulllineitems}
\phantomsection\label{\detokenize{index:pyparsvd.parsvd_base.ParSVD_Base.singular_values}}\pysigline{\sphinxbfcode{\sphinxupquote{property }}\sphinxbfcode{\sphinxupquote{singular\_values}}}
\sphinxAtStartPar
Get the singular values.
\begin{quote}\begin{description}
\item[{Returns}] \leavevmode
\sphinxAtStartPar
singular values.

\item[{Return type}] \leavevmode
\sphinxAtStartPar
ndarray

\end{description}\end{quote}

\end{fulllineitems}


\end{fulllineitems}



\section{ParSVD serial}
\label{\detokenize{index:module-pyparsvd.parsvd_serial}}\label{\detokenize{index:parsvd-serial}}\index{module@\spxentry{module}!pyparsvd.parsvd\_serial@\spxentry{pyparsvd.parsvd\_serial}}\index{pyparsvd.parsvd\_serial@\spxentry{pyparsvd.parsvd\_serial}!module@\spxentry{module}}\index{ParSVD\_Serial (class in pyparsvd.parsvd\_serial)@\spxentry{ParSVD\_Serial}\spxextra{class in pyparsvd.parsvd\_serial}}

\begin{fulllineitems}
\phantomsection\label{\detokenize{index:pyparsvd.parsvd_serial.ParSVD_Serial}}\pysiglinewithargsret{\sphinxbfcode{\sphinxupquote{class }}\sphinxbfcode{\sphinxupquote{ParSVD\_Serial}}}{\emph{\DUrole{n}{K}}, \emph{\DUrole{n}{ff}}, \emph{\DUrole{n}{low\_rank}\DUrole{o}{=}\DUrole{default_value}{False}}, \emph{\DUrole{n}{results\_dir}\DUrole{o}{=}\DUrole{default_value}{\textquotesingle{}results\textquotesingle{}}}}{}
\sphinxAtStartPar
PyParSVD serial class.
\begin{quote}\begin{description}
\item[{Parameters}] \leavevmode\begin{itemize}
\item {} 
\sphinxAtStartPar
\sphinxstyleliteralstrong{\sphinxupquote{K}} (\sphinxhref{https://docs.python.org/2/library/functions.html\#int}{\sphinxstyleliteralemphasis{\sphinxupquote{int}}}) \textendash{} number of modes to truncate.

\item {} 
\sphinxAtStartPar
\sphinxstyleliteralstrong{\sphinxupquote{ff}} (\sphinxhref{https://docs.python.org/2/library/functions.html\#int}{\sphinxstyleliteralemphasis{\sphinxupquote{int}}}) \textendash{} forget factor.

\item {} 
\sphinxAtStartPar
\sphinxstyleliteralstrong{\sphinxupquote{low\_rank}} (\sphinxhref{https://docs.python.org/2/library/functions.html\#bool}{\sphinxstyleliteralemphasis{\sphinxupquote{bool}}}) \textendash{} if True, it uses a low rank algorithm to speed up computations.

\item {} 
\sphinxAtStartPar
\sphinxstyleliteralstrong{\sphinxupquote{results\_dir}} (\sphinxhref{https://docs.python.org/2/library/functions.html\#str}{\sphinxstyleliteralemphasis{\sphinxupquote{str}}}) \textendash{} if specified, it saves the results in \sphinxtitleref{results\_dir}.            Default save path is under a folder called \sphinxtitleref{results} in current working path.

\end{itemize}

\end{description}\end{quote}
\index{incorporate\_data() (ParSVD\_Serial method)@\spxentry{incorporate\_data()}\spxextra{ParSVD\_Serial method}}

\begin{fulllineitems}
\phantomsection\label{\detokenize{index:pyparsvd.parsvd_serial.ParSVD_Serial.incorporate_data}}\pysiglinewithargsret{\sphinxbfcode{\sphinxupquote{incorporate\_data}}}{\emph{\DUrole{n}{A}}}{}
\sphinxAtStartPar
Incorporate new data in a streaming way for SVD computation.
\begin{quote}\begin{description}
\item[{Parameters}] \leavevmode
\sphinxAtStartPar
\sphinxstyleliteralstrong{\sphinxupquote{A}} (\sphinxstyleliteralemphasis{\sphinxupquote{ndarray}}) \textendash{} new data matrix.

\end{description}\end{quote}

\end{fulllineitems}

\index{initialize() (ParSVD\_Serial method)@\spxentry{initialize()}\spxextra{ParSVD\_Serial method}}

\begin{fulllineitems}
\phantomsection\label{\detokenize{index:pyparsvd.parsvd_serial.ParSVD_Serial.initialize}}\pysiglinewithargsret{\sphinxbfcode{\sphinxupquote{initialize}}}{\emph{\DUrole{n}{A}}}{}
\sphinxAtStartPar
Initialize SVD computation with initial data.
\begin{quote}\begin{description}
\item[{Parameters}] \leavevmode
\sphinxAtStartPar
\sphinxstyleliteralstrong{\sphinxupquote{A}} (\sphinxstyleliteralemphasis{\sphinxupquote{ndarray}}) \textendash{} initial data matrix.

\end{description}\end{quote}

\end{fulllineitems}

\index{save() (ParSVD\_Serial method)@\spxentry{save()}\spxextra{ParSVD\_Serial method}}

\begin{fulllineitems}
\phantomsection\label{\detokenize{index:pyparsvd.parsvd_serial.ParSVD_Serial.save}}\pysiglinewithargsret{\sphinxbfcode{\sphinxupquote{save}}}{}{}
\sphinxAtStartPar
Save data.

\end{fulllineitems}


\end{fulllineitems}



\section{ParSVD parallel}
\label{\detokenize{index:module-pyparsvd.parsvd_parallel}}\label{\detokenize{index:parsvd-parallel}}\index{module@\spxentry{module}!pyparsvd.parsvd\_parallel@\spxentry{pyparsvd.parsvd\_parallel}}\index{pyparsvd.parsvd\_parallel@\spxentry{pyparsvd.parsvd\_parallel}!module@\spxentry{module}}\index{ParSVD\_Parallel (class in pyparsvd.parsvd\_parallel)@\spxentry{ParSVD\_Parallel}\spxextra{class in pyparsvd.parsvd\_parallel}}

\begin{fulllineitems}
\phantomsection\label{\detokenize{index:pyparsvd.parsvd_parallel.ParSVD_Parallel}}\pysiglinewithargsret{\sphinxbfcode{\sphinxupquote{class }}\sphinxbfcode{\sphinxupquote{ParSVD\_Parallel}}}{\emph{\DUrole{n}{K}}, \emph{\DUrole{n}{ff}}, \emph{\DUrole{n}{low\_rank}\DUrole{o}{=}\DUrole{default_value}{False}}, \emph{\DUrole{n}{results\_dir}\DUrole{o}{=}\DUrole{default_value}{\textquotesingle{}results\textquotesingle{}}}}{}
\sphinxAtStartPar
PyParSVD parallel class.
\begin{quote}\begin{description}
\item[{Parameters}] \leavevmode\begin{itemize}
\item {} 
\sphinxAtStartPar
\sphinxstyleliteralstrong{\sphinxupquote{K}} (\sphinxhref{https://docs.python.org/2/library/functions.html\#int}{\sphinxstyleliteralemphasis{\sphinxupquote{int}}}) \textendash{} number of modes to truncate.

\item {} 
\sphinxAtStartPar
\sphinxstyleliteralstrong{\sphinxupquote{ff}} (\sphinxhref{https://docs.python.org/2/library/functions.html\#int}{\sphinxstyleliteralemphasis{\sphinxupquote{int}}}) \textendash{} forget factor.

\item {} 
\sphinxAtStartPar
\sphinxstyleliteralstrong{\sphinxupquote{low\_rank}} (\sphinxhref{https://docs.python.org/2/library/functions.html\#bool}{\sphinxstyleliteralemphasis{\sphinxupquote{bool}}}) \textendash{} if True, it uses a low rank algorithm to speed up computations.

\item {} 
\sphinxAtStartPar
\sphinxstyleliteralstrong{\sphinxupquote{results\_dir}} (\sphinxhref{https://docs.python.org/2/library/functions.html\#str}{\sphinxstyleliteralemphasis{\sphinxupquote{str}}}) \textendash{} if specified, it saves the results in \sphinxtitleref{results\_dir}.            Default save path is under a folder called \sphinxtitleref{results} in current working path.

\end{itemize}

\end{description}\end{quote}
\index{incorporate\_data() (ParSVD\_Parallel method)@\spxentry{incorporate\_data()}\spxextra{ParSVD\_Parallel method}}

\begin{fulllineitems}
\phantomsection\label{\detokenize{index:pyparsvd.parsvd_parallel.ParSVD_Parallel.incorporate_data}}\pysiglinewithargsret{\sphinxbfcode{\sphinxupquote{incorporate\_data}}}{\emph{\DUrole{n}{A}}}{}
\sphinxAtStartPar
Incorporate new data in a streaming way for SVD computation.
\begin{quote}\begin{description}
\item[{Parameters}] \leavevmode
\sphinxAtStartPar
\sphinxstyleliteralstrong{\sphinxupquote{A}} (\sphinxstyleliteralemphasis{\sphinxupquote{ndarray}}) \textendash{} new data matrix.

\end{description}\end{quote}

\end{fulllineitems}

\index{initialize() (ParSVD\_Parallel method)@\spxentry{initialize()}\spxextra{ParSVD\_Parallel method}}

\begin{fulllineitems}
\phantomsection\label{\detokenize{index:pyparsvd.parsvd_parallel.ParSVD_Parallel.initialize}}\pysiglinewithargsret{\sphinxbfcode{\sphinxupquote{initialize}}}{\emph{\DUrole{n}{A}}}{}
\sphinxAtStartPar
Initialize SVD computation with initial data.
\begin{quote}\begin{description}
\item[{Parameters}] \leavevmode
\sphinxAtStartPar
\sphinxstyleliteralstrong{\sphinxupquote{A}} (\sphinxstyleliteralemphasis{\sphinxupquote{ndarray}}) \textendash{} initial data matrix.

\end{description}\end{quote}

\end{fulllineitems}

\index{save() (ParSVD\_Parallel method)@\spxentry{save()}\spxextra{ParSVD\_Parallel method}}

\begin{fulllineitems}
\phantomsection\label{\detokenize{index:pyparsvd.parsvd_parallel.ParSVD_Parallel.save}}\pysiglinewithargsret{\sphinxbfcode{\sphinxupquote{save}}}{}{}
\sphinxAtStartPar
Save data.

\end{fulllineitems}


\end{fulllineitems}



\chapter{Postprocessing module}
\label{\detokenize{index:postprocessing-module}}
\sphinxAtStartPar
The postprocessing module is intended to provide some limited support to post\sphinxhyphen{}process
the data and results produced by \sphinxstylestrong{pyparsvd}. The key routines implemented are

\phantomsection\label{\detokenize{index:module-pyparsvd.postprocessing}}\index{module@\spxentry{module}!pyparsvd.postprocessing@\spxentry{pyparsvd.postprocessing}}\index{pyparsvd.postprocessing@\spxentry{pyparsvd.postprocessing}!module@\spxentry{module}}\index{plot\_1D\_modes() (in module pyparsvd.postprocessing)@\spxentry{plot\_1D\_modes()}\spxextra{in module pyparsvd.postprocessing}}

\begin{fulllineitems}
\phantomsection\label{\detokenize{index:pyparsvd.postprocessing.plot_1D_modes}}\pysiglinewithargsret{\sphinxbfcode{\sphinxupquote{plot\_1D\_modes}}}{\emph{\DUrole{n}{modes}}, \emph{\DUrole{n}{idxs}\DUrole{o}{=}\DUrole{default_value}{{[}0{]}}}, \emph{\DUrole{n}{title}\DUrole{o}{=}\DUrole{default_value}{\textquotesingle{}\textquotesingle{}}}, \emph{\DUrole{n}{figsize}\DUrole{o}{=}\DUrole{default_value}{(12, 8)}}, \emph{\DUrole{n}{path}\DUrole{o}{=}\DUrole{default_value}{\textquotesingle{}CWD\textquotesingle{}}}, \emph{\DUrole{n}{filename}\DUrole{o}{=}\DUrole{default_value}{None}}, \emph{\DUrole{n}{rank}\DUrole{o}{=}\DUrole{default_value}{None}}, \emph{\DUrole{n}{value}\DUrole{o}{=}\DUrole{default_value}{\textquotesingle{}abs\textquotesingle{}}}}{}
\sphinxAtStartPar
Plots modes of the SVD decomposition.
\begin{quote}\begin{description}
\item[{Parameters}] \leavevmode\begin{itemize}
\item {} 
\sphinxAtStartPar
\sphinxstyleliteralstrong{\sphinxupquote{modes}} (\sphinxstyleliteralemphasis{\sphinxupquote{ndarray}}) \textendash{} modes.

\item {} 
\sphinxAtStartPar
\sphinxstyleliteralstrong{\sphinxupquote{title}} (\sphinxhref{https://docs.python.org/2/library/functions.html\#str}{\sphinxstyleliteralemphasis{\sphinxupquote{str}}}) \textendash{} if specified, title of the plot.

\item {} 
\sphinxAtStartPar
\sphinxstyleliteralstrong{\sphinxupquote{figsize}} (\sphinxstyleliteralemphasis{\sphinxupquote{tuple}}\sphinxstyleliteralemphasis{\sphinxupquote{(}}\sphinxhref{https://docs.python.org/2/library/functions.html\#int}{\sphinxstyleliteralemphasis{\sphinxupquote{int}}}\sphinxstyleliteralemphasis{\sphinxupquote{,}}\sphinxhref{https://docs.python.org/2/library/functions.html\#int}{\sphinxstyleliteralemphasis{\sphinxupquote{int}}}\sphinxstyleliteralemphasis{\sphinxupquote{)}}) \textendash{} size of the figure               (width,height). Default is (12,8).

\item {} 
\sphinxAtStartPar
\sphinxstyleliteralstrong{\sphinxupquote{path}} (\sphinxhref{https://docs.python.org/2/library/functions.html\#str}{\sphinxstyleliteralemphasis{\sphinxupquote{str}}}) \textendash{} if specified, the plot is saved                at \sphinxtitleref{path}. Default is CWD.

\item {} 
\sphinxAtStartPar
\sphinxstyleliteralstrong{\sphinxupquote{filename}} (\sphinxhref{https://docs.python.org/2/library/functions.html\#str}{\sphinxstyleliteralemphasis{\sphinxupquote{str}}}) \textendash{} if specified, the plot             is saved at \sphinxtitleref{filename}. Default is None.

\item {} 
\sphinxAtStartPar
\sphinxstyleliteralstrong{\sphinxupquote{rank}} (\sphinxstyleliteralemphasis{\sphinxupquote{MPI\_Rank}}) \textendash{} MPI rank for parallel SVD.

\item {} 
\sphinxAtStartPar
\sphinxstyleliteralstrong{\sphinxupquote{value}} (\sphinxhref{https://docs.python.org/2/library/functions.html\#str}{\sphinxstyleliteralemphasis{\sphinxupquote{str}}}) \textendash{} whether to plot absolute              or real value of modes.

\end{itemize}

\end{description}\end{quote}

\end{fulllineitems}

\index{plot\_singular\_values() (in module pyparsvd.postprocessing)@\spxentry{plot\_singular\_values()}\spxextra{in module pyparsvd.postprocessing}}

\begin{fulllineitems}
\phantomsection\label{\detokenize{index:pyparsvd.postprocessing.plot_singular_values}}\pysiglinewithargsret{\sphinxbfcode{\sphinxupquote{plot\_singular\_values}}}{\emph{\DUrole{n}{singular\_values}}, \emph{\DUrole{n}{title}\DUrole{o}{=}\DUrole{default_value}{\textquotesingle{}\textquotesingle{}}}, \emph{\DUrole{n}{figsize}\DUrole{o}{=}\DUrole{default_value}{(12, 8)}}, \emph{\DUrole{n}{path}\DUrole{o}{=}\DUrole{default_value}{\textquotesingle{}CWD\textquotesingle{}}}, \emph{\DUrole{n}{filename}\DUrole{o}{=}\DUrole{default_value}{None}}, \emph{\DUrole{n}{rank}\DUrole{o}{=}\DUrole{default_value}{None}}}{}
\sphinxAtStartPar
Plots singular values of the SVD decomposition.
\begin{quote}\begin{description}
\item[{Parameters}] \leavevmode\begin{itemize}
\item {} 
\sphinxAtStartPar
\sphinxstyleliteralstrong{\sphinxupquote{singular\_values}} (\sphinxstyleliteralemphasis{\sphinxupquote{ndarray}}) \textendash{} singular values.

\item {} 
\sphinxAtStartPar
\sphinxstyleliteralstrong{\sphinxupquote{title}} (\sphinxhref{https://docs.python.org/2/library/functions.html\#str}{\sphinxstyleliteralemphasis{\sphinxupquote{str}}}) \textendash{} if specified, title of the plot.

\item {} 
\sphinxAtStartPar
\sphinxstyleliteralstrong{\sphinxupquote{figsize}} (\sphinxstyleliteralemphasis{\sphinxupquote{tuple}}\sphinxstyleliteralemphasis{\sphinxupquote{(}}\sphinxhref{https://docs.python.org/2/library/functions.html\#int}{\sphinxstyleliteralemphasis{\sphinxupquote{int}}}\sphinxstyleliteralemphasis{\sphinxupquote{,}}\sphinxhref{https://docs.python.org/2/library/functions.html\#int}{\sphinxstyleliteralemphasis{\sphinxupquote{int}}}\sphinxstyleliteralemphasis{\sphinxupquote{)}}) \textendash{} size of the figure (width,height).               Default is (12,8).

\item {} 
\sphinxAtStartPar
\sphinxstyleliteralstrong{\sphinxupquote{path}} (\sphinxhref{https://docs.python.org/2/library/functions.html\#str}{\sphinxstyleliteralemphasis{\sphinxupquote{str}}}) \textendash{} if specified, the plot is saved at \sphinxtitleref{path}.             Default is CWD.

\item {} 
\sphinxAtStartPar
\sphinxstyleliteralstrong{\sphinxupquote{filename}} (\sphinxhref{https://docs.python.org/2/library/functions.html\#str}{\sphinxstyleliteralemphasis{\sphinxupquote{str}}}) \textendash{} if specified, the plot is saved at \sphinxtitleref{filename}.             Default is None.

\item {} 
\sphinxAtStartPar
\sphinxstyleliteralstrong{\sphinxupquote{rank}} (\sphinxstyleliteralemphasis{\sphinxupquote{MPI\_Rank}}) \textendash{} MPI rank for parallel SVD.

\end{itemize}

\end{description}\end{quote}

\end{fulllineitems}



\renewcommand{\indexname}{Python Module Index}
\begin{sphinxtheindex}
\let\bigletter\sphinxstyleindexlettergroup
\bigletter{p}
\item\relax\sphinxstyleindexentry{pyparsvd.parsvd\_base}\sphinxstyleindexpageref{index:\detokenize{module-pyparsvd.parsvd_base}}
\item\relax\sphinxstyleindexentry{pyparsvd.parsvd\_parallel}\sphinxstyleindexpageref{index:\detokenize{module-pyparsvd.parsvd_parallel}}
\item\relax\sphinxstyleindexentry{pyparsvd.parsvd\_serial}\sphinxstyleindexpageref{index:\detokenize{module-pyparsvd.parsvd_serial}}
\item\relax\sphinxstyleindexentry{pyparsvd.postprocessing}\sphinxstyleindexpageref{index:\detokenize{module-pyparsvd.postprocessing}}
\end{sphinxtheindex}

\renewcommand{\indexname}{Index}
\printindex
\end{document}